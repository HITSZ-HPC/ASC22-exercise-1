\documentclass{ctexart}
\usepackage{algorithm}
\usepackage{algorithm2e}
\usepackage[letterpaper,top=2cm,bottom=2cm,left=3cm,right=3cm,marginparwidth=1.75cm]{geometry}
\usepackage[colorlinks=true, allcolors=blue]{hyperref}

\title{HITSZ ASC22 校内练习题}
\author{Easton Man}
\date{November 2021}

\begin{document}

\maketitle

\section{SLIC算法简介}
超像素算法就是将图像中的像素依据某种相似性进行聚类,形成一个大“像素”,这个大“像素”可以作为其他图像处理算法的基础。在众多的超像素算法中,比较常用的是 SLIC(simple linear iterative clustering),它是 Radhakrishna Achanta 等人于 2010 年提出的一种简单高效的超像素算法,它的计算复杂度为 O(N),其中 N 为图像像素点个数。

SLIC 算法的基本思想是, 首先将图像从 RGB 颜色空间转换到 CIE-Lab 颜色空间,并把每个像素的(L,a,b)颜色值和(x,y)坐标值组成一个 5 维的特征向量 V[L, a, b, x, y],然后,根据给定的网格步长 S,初始化聚类中心 Ck=[Lk, ak, bk, xk, yk]T,之后在每个聚类中心 Ck 的邻域(2Sx2S),计算邻域内各像素与该 Ck 点的相似性度量,从而对邻域内的像素点进行聚类,之后迭代更新聚类中心,直至满足收敛条件。

SLIC算法的具体流程如下:
\begin{algorithm}[H]
\caption{SLIC superpixel segmentation}
/∗ Initialization ∗/

Initialize cluster centers Ck = [lk, ak, bk, xk, yk]T by sampling pixels at regular grid steps S.

Move cluster centers to the lowest gradient position in a 3 × 3 neighborhood.

Set label l(i) = −1 for each pixel i. Set distance d(i) = ∞ for each pixel i.

\While {E > threshold}{

/∗ Assignment ∗/

\For {each cluster center Ck} {

    \For {each pixel i in a 2S × 2S region around Ck} {

Compute the distance D between Ck and i.

        \If {D < d(i)}{
            set d(i) = D
            
            set l(i) = k
        }
    }
}

/∗ Update ∗/

Compute new cluster centers.

Compute residual error E.
}

\end{algorithm}



\section{题目说明}
\begin{enumerate}
    \item 源代码包括以下部分:
    \begin{itemize}
        \item main.cpp:入口和计时代码
        \item src/:SLIC源代码
        \item include/:SLIC头文件
        \item data/:数据文件
        \item Makefile:GNU Makefile
    \end{itemize}
    \item 程序运行方法:
    \begin{enumerate}
        \item 编译:make
        \item 运行:make run
    \end{enumerate}
    \item 评价指标:所有测试用例加速比的平均值
    \item 修改范围:
    \begin{itemize}
        \item 允许修改src/
        \item 允许修改include/
        \item 允许新增文件
        \item 允许修改Makefile(包括编译选项和编译器)
        \item 禁止修改main.cpp或任何计时代码
        \item 禁止修改data/中的测试样例
    \end{itemize}
    \item 不允许修改算法的计算量,优化的算法必须和原有算法等价,但可以修改计算顺序等
    \item 正确性判断以基准验证文件check.ppm为准,差异点数为0则通过,其余不通过
    \item 可以针对特定的数据进行优化,但必须保证所有情况下的正确性,后续可能加入更多不同的测试数据
\end{enumerate}

\section{参考文献}

\href{https://core.ac.uk/download/pdf/147983593.pdf}{SLIC Superpixels Compared to State-of-the-art
Superpixel Methods}

\end{document}
